\chapter{Produktleistungen}
	Die Software soll eine Stabilität enthalten, wodurch mehrere Studenten/Dozenten auf einmal darauf zugreifen können. Die Qualitätsanforderung sind Vormaussetzung um ein voll funktionsfähig Programm zu erstellen. Es müssen Schwerpunkte in diesem Programm gelegt werden.\\
	
	Es muss zwischen Funktionalität und Nicht-funktionale Anforderung unterschieden werden. Die Funktionalität ist entscheiden für das Programm. Nicht-funktionale Anforderung sind im ersten Schritt für das Grundprogramm nicht wichtig, können jedoch bei genügend Zeit hinzugefügt werden.\\
	
	Es wurde die ISO / IES Norm 9126 gewählt, um die Softwarequalität sicherzustellen. Diese Norm bezieht sich sehr auf die Qualität der Software als Produkt. Die Produktqualität ist entscheidend, damit das Programm voll funktionsfähig sein wird. 

\chapter{Technische Produktumgebung}
	Auf dem Server wird eine CouchDB-Datenbank verwendet. Auf der Clientseite wird ein Java–Programm installiert.
	
	\section{Software}
	\begin{itemize}
		\item Betriebssystem : Windows 7, Windows 8, Windows 8.1, Mac OS X
		\item Laufzeitsystem: Als Laufzeitsystem wird ein \ac{JRE} benötigt, welches für die meisten Betriebssysteme von Oracle zur Verfügung gestellt wird. \footnote{http://www.oracle.com/technetwork/java/index.html}
		\item Datenbank: Apache CouchDB\footnote{http://couchdb.apache.org}\\
			Apache CouchDB ist ein NoSQL Datenbanksystem von Apache, welche mittels \ac{JSON} die einzelnen Dokumente hinterlegt. Als API stellt CouchDB \ac{HTTP} zur Verfügung. Daher ist CouchDB speziell für Anwendungen entwickelt, welche mit dem Internet verbunden sind und über dieses kommunizieren. CouchDB kann einfach mittels den von Apache bereitgestellten Paketen auf jedem System installiert werden. Es wird Windows, Linux wie auch Mac OSX unterstützt. 
		\item Client: Java Programm
			Als Clientsoftware wird ein Java-Programm benötigt. Es wird Java verwendet, da es auf den meisten Geräten läuft.
	\end{itemize}

	\section{Hardware}
	\begin{itemize}
		\item \underline{Server:} \\
			Die Hardware des Servers muss performant genug sein, um die  	CouchDB lauffähig auf dem System zu halten. Es wird mindestens ein 32-Bit Betriebssystem benötigt.
		
		\item \underline{Client:}\\
			Die Mindestanforderungen für die Clients sind ein lauffähiges \ac{JRE}, welches auf den meisten Betriebssystemen von Oracle zur Verfügung gestellt wird.
	\end{itemize}

	\section{Orgware}
	Der Server muss mit dem Internet verbunden sein. Es wird empfohlen dies über eine synchrone Standleitung (10TBit/s) zu realisieren, um immer verfügbar zu sein.\\
	Die Clients müssen ebenfalls mit dem Internet verbunden sein. Es wird mindestens 2Mbit/s empfohlen.
	
	\section{Produkt-Schnittstellen}
	Anbindung von Java und CoucDB über JSON.

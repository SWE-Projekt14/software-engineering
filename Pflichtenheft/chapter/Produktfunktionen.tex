\chapter{Produktfunktionen}

	% /LF10 Tabelle mit Programmstart/Login
	\begin{table}[H]
		\centering
		\caption{/LF10/Programmstart\_Login}
		\begin{tabularx}{\textwidth}{l|X}
			\toprule
			                        & Programmstart/Login                                         \\ \midrule
			Einstufung              & hoch                                                        \\
			Vorbedingungen          & Installation                                                \\
			Funktion erfolgreich    & Zugang zur Datenbank wird ermöglicht                        \\
			Funktion fehlgeschlagen & -                                                           \\
			Akteure                 & Student, Prüfer/Dozent, Verwaltung                          \\
			Auslöser                & System soll benutzt werden (Noten, Bewertungen, Verwaltung) \\
			Beschreibung            & 1. Benutzer startet das Programm                            \\
			                        & 2. Benutzer gibt Zugangsdaten ein                           \\
			                        & 3. System gibt Rückmeldung ob Login erfolgreich             \\
			Erweiterungen           & -                                                           \\
			Alternativen            & -                                                           \\ \bottomrule
		\end{tabularx}%
		\label{tab:LF10ProLogin}%
	\end{table}%
	
	% /LF20 Feedback erstellen
	\begin{table}[H]
		\centering
		\caption{/LF20/Feedback erstellen}
		\begin{tabularx}{\textwidth}{l|X}
			\toprule
			                        & Feedback erstellen                                                        \\ \midrule
			Einstufung              & mittel                                                                    \\
			Vorbedingungen          & Bewertungen müssen erstellt sein                                          \\
			Funktion erfolgreich    & Feedback kann eingetragen werden                                          \\
			Funktion fehlgeschlagen & Feedback kann nicht eingetragen werden                                    \\
			Akteure                 & Dozenter/Prüfer                                                           \\
			Auslöser                & Feedback ist erwünscht bzw. wird gegeben                                  \\
			Beschreibung            & 1. Prüfer wählt Bewertung aus zu welcher ein Feedback gegeben werden soll \\
			                        & 2. Prüfer trägt Feedback in Textfeld ein                                  \\
			                        & 3. Prüfer schickt Feedback an System ab                                   \\
			Erweiterungen           & -                                                                         \\
			Alternativen            & Persönlich fragen                                                         \\ \bottomrule
		\end{tabularx}%
		\label{tab:LF20createFB}%
	\end{table}%
	
	% /LF30 Feedback abrufen
	\begin{table}[H]
		\centering
		\caption{/LF30/Feedback abrufen}
		\begin{tabularx}{\textwidth}{l|X}
			\toprule
			                        &                           Feedback abrufen \\ \midrule
			             Einstufung &                                     mittel \\
			         Vorbedingungen &                Feedback muss erstellt sein \\
			   Funktion erfolgreich &             Feedback kann abgerufen werden \\
			Funktion fehlgeschlagen &     Feedback kann nicht eingetragen werden \\
			                Akteure &                                    Student \\
			               Auslöser &   Feedback ist erwünscht bzw. wird gegeben \\
			           Beschreibung &        1. Student hat Feedback angefordert \\
			                        &                  2. Feedback wurde gegeben \\
			                        & 3. Student kann gegebenes Feedback abrufen \\
			          Erweiterungen &                                          - \\
			           Alternativen &                          Persönlich fragen \\ \bottomrule
		\end{tabularx}%
		\label{tab:LF30holeFB}%
	\end{table}%
	
	% /LF40 Bewertungsschema erstellen
	\begin{table}[H]
		\centering
		\caption{/LF40/Bewertungsschema erstellen}
		\begin{tabularx}{\textwidth}{l|X}
			\toprule
			                        &               Bewertungsschema erstellen \\ \midrule
			             Einstufung &                                     hoch \\
			         Vorbedingungen &                          Schema erfinden \\
			   Funktion erfolgreich &                  Schema wird gespeichert \\
			Funktion fehlgeschlagen &                                        - \\
			                Akteure &                          Dozente/Prüfer \\
			               Auslöser &     Prüfer will neue Bewertung erstellen \\
			           Beschreibung & 1. Prüfer muss Bewertungskriterien haben \\
			                        &  2. Prüfer trägt Bewertungskriterien ein \\
			                        &            3. Prüfer speichert Schema ab \\
			          Erweiterungen &                                        - \\
			           Alternativen &                        Persönlich fragen \\ \bottomrule
		\end{tabularx}%
		\label{tab:LF40createBewSchem}%
	\end{table}%
	
	% Table generated by Excel2LaTeX from sheet 'Tabelle5'
	\begin{table}[H]
		\centering
		\caption{/LF50/Bewertung abrufen}
		\begin{tabularx}{\textwidth}{l|X}
			\toprule
			                        &                                          Bewertung abrufen \\ \midrule
			             Einstufung &                                                       hoch \\
			         Vorbedingungen &                           Bewertungen müssen erstellt sein \\
			   Funktion erfolgreich &                                   Bewertung wird abgerufen \\
			Funktion fehlgeschlagen &                                                          - \\
			                Akteure &                                        Student, Verwaltung \\
			               Auslöser &                      Bewertung für Student wurde abgegeben \\
			           Beschreibung & 1.Student bekommt Information dass Bewertung verfügbar ist \\
			                        &                               2. Student ruft Bewertung ab \\
			          Erweiterungen &                                                          - \\
			           Alternativen &                                          Persönlich fragen \\ \bottomrule
		\end{tabularx}%
		\label{tab:LF50holeBew}%
	\end{table}%
	
	% Bewertung eintragen (H2)
	\begin{table}[H]
		\centering
		\caption{/LF60/Bewertung eintragen (H2)}
		\begin{tabularx}{\textwidth}{l|X}
			\toprule
			                        &                     Bewertung eintragen (H2) \\ \midrule
			             Einstufung &                                         hoch \\
			         Vorbedingungen &                    Schema muss erstellt sein \\
			   Funktion erfolgreich &            Bewertung wird Schema hinzugefügt \\
			Funktion fehlgeschlagen &            Automatisch schlechtere Bewertung \\
			                Akteure &                              Dozenter/Prüfer \\
			               Auslöser &  Prüfer Bewertet einzelne Punkte des Schemas \\
			           Beschreibung &        1. Prüfer wählt erstelltes Schema aus \\
			                        &         2. Prüfer bewertet nach H2 Kriterium \\
			                        & 3. Schema mit Bewertungen wird abgespeichert \\
			          Erweiterungen &                                            - \\
			           Alternativen &                                            - \\ \bottomrule
		\end{tabularx}%
		\label{tab:LF60eintrBew}%
	\end{table}%
	
	\begin{table}[H]
		\centering
		\caption{/LF70/Score S2R}
		\begin{tabularx}{\textwidth}{l|X}
			\toprule
			                        & Score S2R                                        \\ \midrule
			Einstufung              & hoch                                             \\
			Vorbedingungen          & H2 ist vorhanden                                 \\
			Funktion erfolgreich    & Berechnung ist erfolgreich                       \\
			Funktion fehlgeschlagen & -                                                \\
			Akteure                 & Dozenter/Prüfer                                  \\
			Auslöser                & Prüfer will Bewertung ausrechnen                 \\
			Beschreibung            & 1. Prüfer muss H2 Bewertung einstellen           \\
			                        & 2. Das Programm berechnet eine Rate              \\
			                        & 3. Rate wird abgespeichert                       \\
			Erweiterungen           & Weitere Parameters / Impacts (R2S) können folgen \\
			Alternativen            & -                                                \\ \bottomrule
		\end{tabularx}%
		\label{tab:LF70S2R}%
	\end{table}%

	% Score R2S
	\begin{table}[H]
		\centering
		\caption{/LF80/Score R2S }
		\begin{tabularx}{\textwidth}{l|X}
			\toprule
			                        & Score R2S                                                            \\ \midrule
			Einstufung              & hoch                                                                 \\
			Vorbedingungen          & S2R ist vorhanden                                                    \\
			Funktion erfolgreich    & Weitere Parameters / Impacts sollen in den Score eingerechnet werden \\
			Funktion fehlgeschlagen & -                                                                    \\
			Akteure                 & Dozenter/Prüfer                                                      \\
			Auslöser                & Prüfer will weitere Parameters einbinden                             \\
			Beschreibung            & 1. Prüfer muss S2R besitzen                                          \\
			                        & 2. Das Programm berechnet weite Parameters / Impacts                 \\
			                        & 3. Neue Rate wird abgespeichert                                      \\
			Erweiterungen           & -                                                                    \\
			Alternativen            & -                                                                    \\ \bottomrule
		\end{tabularx}%
		\label{tab:LF80R2S}%
	\end{table}%
\chapter{Produkteinsatz}
	Der geplante Einsatz des Systems sind die Qualitätsanforderungen.
	
	\section{Anwendungsbereiche}
	Das System soll vor allem zur Bewertung von Schülern und Studenten dienen. Es können jedoch auch andere Institutionen oder Firmen, welche ein einheitliches Bewertungssystem wollen, welches eine faire Bewertung ermöglicht. Es muss jedoch garantiert werden, dass keine fremden Personen Zugriff auf das System bekommen, da es sich bei solchen Bewertungen um teils sensible Informationen handelt. Studenten sollten die einfache Möglichkeit besitzen nachdem sie einen Test durchlaufen haben ihre Bewertung zu erfahren und diese auf einer einfachen Oberfläche teils grafisch dargestellt werden soll um einen Vergleich zu ermöglichen.
	
	\section{Zielgruppen}
	\begin{itemize}
		\item \underline{Prüfer:}\\
	Der Prüfer ist der Ersteller sowie der Bearbeiter der Prüfung. Dieser muss sich sinnvolle Bewertungskriterien überlegen, sowie in das System eintragen. Der Prüfer benötigt alle Funktionen zum Eintragen der Bewertungen bzw. der Scores.
	
		\item \underline{Verwalter (Admin):}\\
	Der Verwalter besitzt die Rechte zur Erstellung sowie Löschung der Benutzer. Zusätzlich kann er Benutzergruppe erstellen sowie die Benutzer in die jeweiligen Benutzergruppen zuordnen. Entweder ist der Benutzer ein Student oder Dozent / Prüfer. Der Verwalter kann zusätzlich die Bewertungen verwalten.
			
		\item \underline{Verwaltung:}\\
	Die Verwaltung benötigt Zugriff auf das System um die Score auslesen zu können. Diese werden in einem Archiv abgespeichert und für 10 Jahre hinterlegt. 	
	
	\item \underline{Dozent:}\\
	Der Dozent besitzt die gleichen Rechte wie der Prüfer.
		
	\end{itemize}

	\section{Betriebsbedingungen}
	
	\begin{itemize}
	\item \underline{Physikalische Umgebung:}\\
	Die Software soll später auf jedem Computer lauffähig sein. Die Datenbank mit API wird auf einem Server installiert. Somit besteht die Möglichkeit, dass mehrere Personen gleichzeitig Zugriff auf die Daten haben. Über ein kleines Programm kann der Prüfer sich mit der Datenbank verbinden und darauf zuzugreifen.
	
	\item \underline{Tägliche Betriebszeit:} \\
	Der Server auf dem die Daten hinterlegt sind muss immer verfügbar sein, damit die Prüfer wie auch die Prüflinge jederzeit mit der Anwendung Zugriff haben.
	Ein reduziertes Programm (Prototyp) wird vor dem eigentlichen Programm entwickelt.
		
	\item \underline{Betrieb:} \\
	Es ist ein unbeaufsichtigter Betrieb vorgesehen. Sobald der Server einmal installiert ist, sollte das System ohne weitere Konfigurationen funktionieren. Der Prüfer sollte sich nicht mit der Konfiguration des System belasten müssten. 	
	\end{itemize}

\chapter{Qualitätsanforderungen}

%Qualitätsanforderungen	
\begin{table}[H]
	\centering
	\caption{Qualitätsanforderungen}
	\begin{tabularx}{\textwidth}{l|X|X|X|X}
		\toprule
		\textbf{Produktqualität}       & \centering \textbf{sehr gu}t & \centering \textbf{gut}      & \centering \textbf{normal}   & \textbf{irrelevant} \\
		\toprule
		\multicolumn{5}{l}{\textbf{Funktionalität}}                           \\ 
		\midrule
		\quad Angemessenheit        &          &          & $\times$ &  \\ \hline
		\quad Richtigkeit           &          &  $\times$ &          &  \\ \hline
		\quad Interoperabilität     &          &          & $\times$ &  \\ \hline
		\quad Ordnungsmäßigkeit     &          & $\times$ &          &  \\ \hline
		\quad Sicherheit            &          & $\times$ &          &  \\
		\toprule
		\multicolumn{5}{l}{\textbf{Zuverlässigkeit}}                          \\
		\midrule
		\quad Reife                 &          & $\times$ &          &  \\ \hline
		\quad Fehlertolleranz       &          &          & $\times$ &  \\ \hline
		\quad Wiederherstellbarkeit &          &          & $\times$ &  \\ 
		\toprule
		\multicolumn{5}{l}{\textbf{Benutzbarkeit}}                            \\ 
		\midrule
		\quad Verständlichkeit      & $\times$ &          &          &  \\ \hline
		\quad Erlernbarkeit         & $\times$ &          &          &  \\ \hline
		\quad Bedienbarkeit         & $\times$ &          &          &  \\ 
		\toprule
		\multicolumn{5}{l}{\textbf{Effizienz}}                                \\
		\midrule
		\quad Zeitverhalten         &          &          &          & $\times$       \\ \hline
		\quad Verbrauchsverhalten   &          &          & $\times$ &  \\ 
		\toprule
		\multicolumn{5}{l}{\textbf{Änderbarkeit}}                             \\ 
		\midrule
		\quad Analysierbarkeit      &          &          & $\times$ &  \\ \hline
		\quad Modifizierbarkeit     &          &          &          & $\times$       \\ \hline
		\quad Stabilität            &          &          & \centering  $\times$ &  \\ \hline
		\quad Prüfbarkeit           &          &          & \centering $\times$ &  \\ 
		\toprule
		\multicolumn{5}{l}{\textbf{Übertragbarkeit}}                          \\ 
		\midrule
		\quad Anpassbarkeit         &          &          &          & $\times$       \\ \hline
		\quad Installierbarkeit     &          &          &          & $\times$       \\ \hline
		\quad Konformität           &          &          & $\times$ &  \\ \hline
		\quad Austauschbarkeit      &          &          &          & $\times$       \\ \hline
	\end{tabularx}%
	\label{tab:QualAnfor}%
\end{table}%
	
	\begin{itemize}
		\item Funktionalität:\\
		In der Rubrik Funktionalität spielt vor allem die Sicherheit eine große Rolle, weshalb die Qualitätsanforderungen hier sehr gut sein müssen. Es muss garantiert werden, dass keine Daten geklaut oder missbraucht werden können. Deshalb ist es ratsam hier mehr Zeit zu investieren. Die Punkte Richtigkeit und Ordnungsmäßigkeit müssen qualitativ nicht so hochwertig sein wie die Sicherheit, weil kleine Pannen bei der Benutzung keine gravierenden Schäden anrichten. Normale Qualität genügt bei Angemessenheit und Interoperabilität.\\
		
		\item Zuverlässigkeit:\\
		Insgesamt sollte das Programm zuverlässig laufen, jedoch müssen nicht überwiegend Ressourcen dafür verbraucht werden. Sollte das System abstürzen oder nicht ordnungsgemäß laufen, kann man nach einem Neustart weiterarbeiten.\\
		
		\item Benutzbarkeit:\\
		Viel Augenmerk wird auf die Benutzbarkeit gelegt. Das Programm sollte für den Anwender einfach zu bedienen sein und sich selbst erklären, damit keine Schulung mehr notwendig ist. \\
		
		\item 	Effizienz:\\
		Ausschlaggebend ist die Effizienz nicht, da das Bewertungssystem nicht besonders schnell die Score abrufen und verarbeiten muss. \\
		
		\item Änderbarkeit:		
		Für die Zukunft können weitere Features eingebaut werden, wenn man möchte. Grundsteine für weitere Features werden nicht gezielt gelegt.\\
					
		\item  Übertragbarkeit:
		Die Übertragbarkeit spielt eine untergeordnete Rolle, da man nur drauf zugreifen muss. Installieren oder ähnliches muss nicht getan werden. \\
			
	\end{itemize}